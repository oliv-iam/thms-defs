\noindent
\textbf{\LARGE Definitions and Theorems} \\
\large Week: Nov. 4 - Nov. 8

\normalsize

\section*{Theorems, Propositions and Corollaries}

\subsection*{2.5 Composition of Linear Transformations}

\textbf{Prop 2.5.1:} Let $S: U \rightarrow V, \; T: V \rightarrow W$ be linear maps. Then $TS: U \rightarrow W$ is linear.

\bigskip

\noindent
\textbf{Prop:} Let $S: V \rightarrow W, \; T: V \rightarrow W$ be linear maps. Then the following functions are also linear. 
\begin{enumerate}
    \item $S + T: V \rightarrow W, \; (S + T)(\bar x) := S(\bar x) + T(\bar x).$
    \item if $c \in \mathbb{R}$ is fixed, $cT: V \rightarrow W,\; (cT)(\bar x) := c \cdot T(\bar x)$.
\end{enumerate}

\bigskip 

\noindent 
\textbf{Prop 2.5.4:} 
Let
$\begin{aligned}
    R, R_1, R_2 &: U \rightarrow V \\
    S, S_1, S_2 &: V \rightarrow W \\ 
    T &: W \rightarrow Z
\end{aligned}$ be linear maps. Then 
\begin{enumerate}
    \item $T(SR) = (TS)R$
    \item $S(R_1 + R_2) = SR_1 + SR_2$
    \item $(S_1 + S_2)R = S_1R + S_2R$
\end{enumerate}

\bigskip 

\noindent 
\textbf{Prop 2.5.6:} Let $S: U \rightarrow V$ and $T: V \rightarrow W$ be linear maps. Then 
\begin{enumerate}
    \item ker($S$) $\subseteq$ ker($TS$)
    \item im($TS$) $\subseteq$ im($T$)
\end{enumerate}

\bigskip 

\noindent 
\textbf{Thm 2.5.13:} Let $S: U \rightarrow V, \; T: V \rightarrow W$ be linear maps and $\alpha, \beta, \gamma$ be finite bases for $U, V, W$ respectively. 
Then $[TS]_\alpha^\beta = [T]_\beta^\gamma[S]_\alpha^\beta$. 
So, $\forall \bar v \in V$, $[(TS)(\bar v)]_\gamma = [T]_\beta^\gamma[S]_\alpha^\beta[\bar v]_\alpha$

\bigskip 

\noindent 
\textbf{Corollary 2.5.14:} Let $A, B, C$ be matrices of the "correct sizes" (so that matrix multiplication is well-defined). Then 
\begin{enumerate}
    \item $(AB)C = A(BC)$
    \item $A(B + C) = AB + AC$
    \item $(A + B)C = AC + BC$
\end{enumerate}

\bigskip 
\noindent
\textbf{Prop:} Let $A \in M_{m \times n}(\mathbb{R})$. Then 
\begin{enumerate}
    \item $AI_{n \times n} = A$ 
    \item $I_{m \times m}A = A$
\end{enumerate}

\subsection*{2.6 Inverse of a Linear Transformation}

\textbf{Prop:} Let $T: V \rightarrow W$ be an invertible linear map. (so $T^{-1}: W \rightarrow V$ exists).
\begin{enumerate}
    \item $T^{-1}$ is unique 
    \item $T^{-1}: W \rightarrow V$ is also invertible and $(T^{-1})^{-1} = T$
    \item If $R: U \rightarrow V$ is invertible then $TR: U \rightarrow W$ is invertible and $(TR)^{-1} = R^{-1}T^{-1}$
\end{enumerate}

\bigskip 

\noindent 
\textbf{Thm 2.6.2:} A linear map is invertible iff it is bijective. 

\bigskip 

\noindent 
\textbf{Thm 2.6.1:} Let $T: V \rightarrow W$ be an invertible linear map. Then $T^{-1}:W \rightarrow V$ is linear. 

\bigskip 

\noindent
\textbf{Prop 2.6.7:} Let $V$ and $W$ be vector spaces where dim($V$) $<$ $\infty$. \\ 
Then $V \cong W$ iff dim($V$) = dim($W$). 

\bigskip 

\noindent 
\textbf{Corollary:} All $n$-dim vector spaces are isomorphic to $\mathbb{R}^n$. 

\pagebreak

\section*{Definitions} 

\textbf{Def:} Let $U,V,W$ be vector spaces and $S: U \rightarrow V, \; T: V \rightarrow W$ be two linear maps. 
The \textit{composition} of $S$ with $T$ (written $T \circ S$ or $TS$) is the function $TS: V \rightarrow W$ defined by 
\[(TS)(\bar u) = T(S(\bar u)), \; \forall \bar u \in U\].

\noindent 
\textbf{Def 2.5.10:} $A := [a_{ij}] \in M_{n \times r}(\mathbb{R}), \; B := [b_{ij}] \in M_{m \times n}(\mathbb{R})$. \\
Then the \textit{matrix product} \\ 
\[BA \in M_{m \times n}(\mathbb{R})\]  
 is the matrix whose $j$th entry is 
\[b_{i1}a_{1j} + \ldots + b_{in}a_{nj} = \sum_{k = 1}^{n} b_{ik}a_{kj} \] 

\bigskip 

\noindent
\textbf{Def:} A linear map $T: V \rightarrow W$ is said to be \textit{invertible} if there is a function $S: W \rightarrow V$ (called the \textit{inverse of }$T$) satisfying 
\begin{enumerate}
    \item $(ST)(\bar v) = \bar v, \; \forall \bar v \in V$
    \item $(TS)(\bar w) = \bar w, \; \forall \bar w \in W$
\end{enumerate}

\bigskip 

\noindent 
\textbf{Def 2.6.3/2.6.4:} 
\begin{enumerate}
    \item An invertible linear map is called an \textit{isomorphism}. 
    \item Two vector spaces $V$ and $W$ are called \textit{isomorphic} (written as $V \cong W$) if there exists an isomorphism $T: V \rightarrow W$.
\end{enumerate}