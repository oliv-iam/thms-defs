\noindent
\textbf{\LARGE Definitions and Theorems} \\
\large Week: Dec. 2 - Dec. 4

\normalsize

\section*{Theorems and Propositions}
\subsection*{4.1 Eigenvectors and Eigenvalues}
\textbf{Prop 4.1.9+:} Let $A \in M_{n \times n}(\mathbb{R})$. Then $\lambda \in \mathbb{R}$ is an eigenvalue of $A$ iff $\det(A-\lambda I)=0 \in \mathbb{R}$. In this case, every nonzero vector in $\ker(A-\lambda I)$ is an eigenvector of $A$ with the eigenvalue $\lambda$.

\bigskip 

\noindent 
\textbf{Prop:} Let $A, B \in M_{n \times n}(\mathbb{R})$. Suppose $A$ and $B$ are similar. Then $P_A = P_B$.

\bigskip 

\noindent 
\textbf{Prop:} Let $T: V \rightarrow V$ be a linear map, where $\dim(V) < \infty$, and let $\lambda \in \mathbb{R}$. Then the following are equivalent:
\begin{enumerate}
    \item $\lambda$ is an eigenvalue of $T$ 
    \item $\lambda$ is an eigenvalue of the $n \times n$ matrix $[T]_\alpha^\alpha$
    \item $\det([T]_\alpha^\alpha-\lambda I)=0$ 
\end{enumerate}
Moreover, a vector $\bar v \in V$ is an eigenvector of $T$ iff $[\bar v]_\alpha \in \mathbb{R}^n$ is an eigenvector of $[T]_\alpha^\alpha$.

\subsection*{4.2 Diagonalizability}
\textbf{Thm:} Let $T: V \rightarrow V$ be a linear map, where $\dim(V) < \infty$. Then the following are equivalent:
\begin{enumerate}
    \item $T$ is diagonalizable 
    \item There is a basis $\alpha$ of $V$ consisting entirely of eigenvectors of $T$ 
    \item For any basis $\beta$ of $V$, the matrix $[T]_\beta^\beta$ is similar to a diagonal matrix, i.e., $\exists Q \in M_{n \times n}(\mathbb{R})$ which is invertible and $\exists D \in M_{n \times n}(\mathbb{R})$ which is diagonal such that $[T]_\beta^\beta=QDQ^{-1}$.
\end{enumerate}
In this case, the diagonal entries of $D$ are then eigenvalues of $T$(?) and the columns of $Q$ are the coordinate vectors corresponding to the eigenvectors of $T$.

\bigskip 

\noindent 
\textbf{Corollary:} Let $A \in M_{n \times n}(\mathbb{R})$. Then the following are equivalent:
\begin{enumerate}
    \item A is diagonalizable (i.e., $T_A:\mathbb{R}^n \rightarrow \mathbb{R}^n$, defined $\bar x \mapsto A\bar x$, is diagonalizable)
    \item There is a basis of $\mathbb{R}^n$ consisting entirely of eigenvectors of $A$ 
    \item $A$ is similar to a diagonal matrix $D \in M_{n \times n}(\mathbb{R})$, i.e., $\exists Q \in M_{n \times n}(\mathbb{R})$ which is invertible and satisfies $A=QDQ^-1$
\end{enumerate}
In this case, the diagonal enries of $D$ are the eigenvalues of $A$ and the columns of $Q$ are the corresponding eigenvectors of $A$.

\pagebreak 

\section*{Definitions}
\textbf{Def 4.1.2:} Let $T: V \rightarrow V$ be a linear map. A vector $\bar v \in V$ is said to be an \textit{eigenvector} of $T$ if $\bar v \ne \bar 0$ and $\exists \lambda \in \mathbb{R}$ such that $T(\bar v)=\lambda \bar v$. In this case, $\lambda$ is called an \textit{eigenvalue} of $T$ corresponding to eigenvector $\bar v$.

\bigskip 

\noindent 
\textbf{Def:} Let $A \in M_{n \times n}(\mathbb{R})$. The polynomial $P_A:\mathbb{R} \rightarrow \mathbb{R}$ given by $p_A(t):=\det(A-tI)$ is called the \textit{characteristic polynomial} of $A$.

\bigskip 

\noindent 
\textbf{Def:} Let $T: V \rightarrow V$ be a linear map, where $\dim(V) < \infty$. The map $T$ is said to be \textit{diagonalizable} if there exists a basis $\alpha$ of $V$ such that $[T]_\alpha^\alpha$ is a diagonal matrix, i.e., 
\[[T]_\alpha^\alpha = 
\begin{bmatrix}
    \lambda_1 & & 0 \\ 
     & \ddots \\
    0 & & \lambda_n
\end{bmatrix} \]
for some $\lambda_i \in \mathbb{R}$.