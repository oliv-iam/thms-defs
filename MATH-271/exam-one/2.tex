\noindent
\textbf{\LARGE Definitions and Theorems} \\
\Large Week: Sep. 9 - 13

\section{Theorems}

\textbf{Thm 1.2.8:} Let $V$ be a vector space. Then $W$ is a subspace of $V$ if and only if 
\begin{enumerate}
    \item $W \subseteq V$
    \item $\bar 0_v \in W$
    \item $\forall \bar x, \bar y \in W, \; \forall c \in \mathbb{R}, \; c \bar x + \bar y \in W$
\end{enumerate}

\bigskip
\noindent
\textbf{Thm:} Let $V$ be a vector space and $W \subseteq V$. Then $W$ is a subspace of $V$ if and only if the following three statements hold true:
\begin{enumerate}
    \item $\bar 0_v \in W$
    \item $W$ is closed under addition
    \item $W$ is closed under scalar multiplication
\end{enumerate}

\bigskip
\noindent
\textbf{Thm:} Let $V$ be a vector space. 
\begin{enumerate}
    \item $V$ is a subspace of $V$
    \item {$\bar 0_v$} is a subspace of $V$.
\end{enumerate}

\pagebreak
\section{Definitions}

\textbf{Def 1.2.6:} Let $V$ be a vector space. A set $W$ is called a \textit{subspace} of $V$ if
\begin{enumerate}
    \item $W \subseteq V$
    \item $W$ is a vector space under the same operations as in $V$.
\end{enumerate}

\bigskip
\noindent
\textbf{Defining Sets}

% \noindent
\textbf{Def:} Let $A$ and $B$ be two sets. Then $A$ is a \textit{subset} of $B$ if $\forall x \in A, \; x \in B$. 

% \noindent
\textbf{Def:} Two sets $A$ and $B$ are equal if $A \subseteq B$ and $B \subseteq A$.

\bigskip
\noindent
\textbf{Def 1.3.1:} Let $S$ be a subset of a vector space $V$.
\begin{itemize}
    \item[a.] A (finite) \textit{linear combination} of vectors is any vector of the form $c_1 \bar v_1 + c_2 \bar v_2 + ... + c_n \bar v_n$, where $ c_i \in \mathbb{R}$ and $\bar v_i \in S$.
    \item[b.] If $S \neq \varnothing $, then the set of all possible linear combinations of vectors in $S$ is called the (linear) \textit{span} of $S$, and is denoted Span($S$). 
    \newline If $ S = \varnothing$ , then define Span($S$) = {$\bar 0_V$}
    \item[c.] Let $W \subseteq V$ be a subset. If $W = $Span($S$), the $S$ is said to \textit{span} (or \textit{generate}) $W$. If $S$ is a finite set, then $W$ is said to be finitely generated.
\end{itemize}

\pagebreak
\section{Propositions}

\textbf{Prop 1.1.1:} Let V be a vector space. Then
\begin{enumerate}
    \item The vector $\bar z $ in A3 is unique
    \item $\forall x \in V, \; 0 \cdot \bar x = \bar 0 v$
    \item The vector $\bar y$ in A4 is unique.
    \item $\forall \bar x \in V, \; (-1) \cdot \bar x = -\bar x.$
\end{enumerate}