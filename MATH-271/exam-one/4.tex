\noindent
\textbf{\LARGE Definitions and Theorems} \\
\Large Week: Sep. 23 - 27

\section{Theorems}

\textbf{Thm:} Let $S$ be a subset of a vector space $V$. Then $S$ is linearly dependent iff $\exists \bar v \in S$ s.t. $\bar v$ is a linear combination of other vectors in $S$. In this case, Span($S$) = Span($s \backslash \{\bar v \}$).

\bigskip

\noindent
\textbf{Thm 1.5:} A homogenous system of equations with $m$ equations and $n$ unknowns has a \textit{nontrivial} solutions if $m < n$.

\section{Corollaries}

\textbf{Corollary:} Let $S := \{\bar v_1, \ldots, \bar v_n \} \subseteq \mathbb{R}^m$. If $n > m$ then $S$ is linearly dependent.

\pagebreak
\section{Propositions}

\textbf{Prop 1.4.7:} Let $V$ be a vector space and $S, T$ be subsets of $V$, and assume $S \subseteq T$. Then 
\begin{enumerate}
    \item If $S$ is linearly dependent, then $T$ is linearly dependent.
    \item If $T$ is linearly independent, then $S$ is linearly independent.
\end{enumerate}

\bigskip

\noindent
\textbf{Prop 1.5.3:} Given a system ($\star$), the new system obtained by performing any of the row operations 1 - 3 is equivalent to ($\star$), i.e., ($\star$) and the new system have the same solution set. \\
Elementary Row Operations:
\begin{enumerate}
    \item \textit{Row exchange}: interchange any two equations in the system.
    \item \textit{Scaling}: multiply any single equation by a nonzero number.
    \item \textit{Row replacement}: replace an equation by its sum with a constant multiple of another equation.
\end{enumerate}

\pagebreak
\section{Definitions}

\textbf{Def:} A system of $m$ equations and $m$ unknowns (variables) $x_1, \ldots , x_2$ having the form
\[
(\star)
\left\{ 
    \begin{array}{l}
        a_{11}x_1 + a_{12}x_2 + \ldots + a_{1n}x_n = b_1 \\
        a_{21}x_1 + a_{22}x_2 + \ldots + a_{2n}x_n = b_2 \\
        \vdots \\
        a_{m1}x_1 + a_{m2}x_2 + \ldots + a_{mn}x_n = b_m,
    \end{array} 
\right.
\]
where $a_{ij}, b_i \in \mathbb{R}$ (coefficients) is called a \textit{system of linear equations}. The system ($\star$) is called \textit{homogenous} if $b_1 = b_2 = \ldots = 0$. Otherwise, it is called \textit{inhomogeous}. 

\bigskip

\noindent
\textbf{Def:} A vector $\bar c = (c_1, \ldots , c_n) \in \mathbb{R}^n$ is called a solution to ($\star$) if all equations in ($\star$) are satisfied with $x_1 = c_1, x_2 = c_2, \ldots , x_n = c_n$. The set \{$\bar c \in \mathbb{R}^n \mid \bar c$ is a solution to ($\star$)\} is called the \textit{solution set} for ($\star$).

\bigskip

\noindent
\textbf{Def:} A system of linear equations is in \textit{echelon form} if it has all the following properties:
\begin{enumerate}
    \item In each equation, all coefficients are zero or the \textit{leading term}, i.e., the first nonzero coefficient (read left to right), is 1. 
    \item For any nonzero equation, its leading term is strictly to the right of the leading term in the equations above it.
    \item All entries above and below each leading term are zero (or blank). 
\end{enumerate}
Variables in leading term are called \textit{basic} (or \textit{pivot}) \textit{variables}. All other variables are \textit{free}.

\bigskip

\noindent
\textbf{Def 1.6.1:} Let $V$ be a vector space. A set $S \subseteq V$ is said to be a \textit{basis} for $V$ if both 
\begin{enumerate}
    \item $S$ spans $V$ (so Span($S$) = $V$)
    \item $S$ is linearly independent.
\end{enumerate}
