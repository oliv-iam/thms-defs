\noindent
\textbf{\LARGE Definitions and Theorems} \\
\large 2-5: Basic properties of groups

\newcommand{\congmod}[3]{
        #1 \equiv #2 \text{ (mod } #3)
    }

\normalsize

\section*{Definitions}

\subsection*{Groups, subgroups}

\defn $\ast$ is a \textit{binary operation} on a set $G$ if $\forall x, y \in G,\; x \ast y \in G$

\defn Let $G$ be a set, and let $\ast: G \times G \rightarrow G$ be a \textit{binary operation} on $G$. Suppose
\begin{enumerate}
    \item $\ast$ is \textit{associative}, i.e., $\forall x, y \in G, \; (x \ast y) \ast z = x \ast (y \ast z)$
    \item $\ast$ has an \textit{identity element}, i.e., $\exists e \in G$ s.t. $\forall x \in G, \; x \ast e = e \ast x = x$
    \item $\ast$ has \textit{inverses}, i.e., $\forall x \in G, \; \exists y \in G $ s.t. $x \ast y = y \ast x = e$
\end{enumerate}
We then say that $(G, \ast)$ is a \textit{group}.
If $(G, \ast)$ is a group and $\ast$ is \textit{commutative}, we say $G$ is an \textit{abelian group}.

\defn Let $G$ be a group. Suppose $\exists a \in G$ s.t. $G = \{a^m \mid m \in \Z\}$, i.e., $\exists a \in G$ s.t. $\forall g \in G, \; \exists m \in \Z $ s.t. $g = a^m$. 
Then we say 
\begin{itemize}
    \item $a$ is a \textit{generator}
    \item $G$ is \textit{cyclic}
\end{itemize}

\defn Let $G$ be a group. 
\begin{itemize}
    \item the \textit{order of} $G$, denoted $|G|$ or $\#G$, is the number of elements in $G$ 
    \item let $a \in G$. The \textit{order of} $a$, denoted $o(a)$, is the smallest integer $n \ge 1$ s.t. $a^n = e$, or $\infty$ if no such $n$ exists
\end{itemize}

\defn Let $(G, \ast)$ be a group, with subset $H \subseteq G$. If $(H, \ast)$ is also a group, we say $H$ is a \textit{subgroup} of $G$.

\defn Let $G$ be a group. 
\begin{itemize}
    \item The \textit{center} of $G$ is $Z(G) = \{z \in G \mid zg = gz \forall g \in G\}$
    \item Let $g \in G$. The \textit{centralizer} of $g$ is $Z(g) = \{z \in G \mid zg = gz \}$
\end{itemize}

\subsection*{Related definitions}

\defn Let $m, n \in \Z$. We say $n \textit{ divides } m$, written $n \mid m$ if $\exists q \in \Z$ s.t. $m = nq$

\defn Let $m, n \in \Z$, not both $0$. The \textit{greatest common divisor} of $m, n$, denoted $\gcd(m, n)$ or $(m, n)$ is the largest integer $d \ge 1$ s.t. $d \mid m$ and $d \mid n$.

\defn $m, n \in \Z$ are called \textit{relatively prime} (or \textit{coprime}) if $(m, n) = 1$. 

\bigskip 

\noindent
Common Sets:
\begin{itemize}
    \item $\Z$: set of integers 
    \item $GL(2, \R)$: $2 \times 2$ invertible matrices with $\R$ entries
    \item $\mathbb{Q}$: rational numbers (can be expressed as a fraction)
    \item $C_n = \{j \in \Z \mid 0 \le j \le n - 1 \}$
\end{itemize}

\bigskip

\noindent
Misc (operations, notation, etc.):
\begin{itemize}
    \item $\oplus$: for any $a, b \in C, \; a \oplus b = 
    \begin{cases}
        a + b & \text{if } a + b < n \\
        a + b - n & \text{if } a + b \ge n
    \end{cases}$
    \item $\congmod{j}{k}{n}$ means $n \mid (j - k)$
    \item $\langle a \rangle = \{a^n \mid n \in \Z\}$
\end{itemize}

\pagebreak

\section*{Theorems}

Findings from homework:
\begin{itemize}
    \item $G$ a group. $(G, \ast)$ is abelian iff $(x \ast y)^{-1} = x^{-1} \ast y^{-1} $ for all $x, y \in G$
    \item Let $G$ be a group s.t. $x^2 = e$ for all $x \in G$. Then $G$ is abelian
    \item Let $G$ be a group. Then $G$ is abelian iff $(x \ast y)^2 = x^2 \ast y^2 $ for all $x, y \in G$
    \item Fix $n \ge 1$. Let $x_1, x_2, y_1, y_2 \in \Z$ be integers s.t. $\congmod{x_1}{x_2}{n}$ and $\congmod{y_1}{y_2}{n}$. Then $\congmod{x_1y_1}{x_2y_2}{n}$
    \item Let $G$ be a group, with $a, x \in G$. Then for any integer $n \ge 1$, we have \\ $(xax^{-1})^n = xa^nx^{-1}$ 
    \item Every subgroup of an abelian group is abelian
\end{itemize}

\bigskip

\noindent
\textbf{Lemma:} (division algorithm) Let $a, n \in \Z$ with $ n \ge 1$. Then $\exists ! q, r \in \Z$ s.t.
\begin{enumerate}
    \item $a = qn + r$
    \item $0 \le r \le n - 1$
\end{enumerate}

\prop Let $G$ be a group.
\begin{enumerate}
    \item The identity element $e$ is unique
    \item $\forall x \in G$, the inverse $x^{-1}$ is unique
\end{enumerate}

\prop Let $G$ be a group. Then 
\begin{enumerate}
    \item $\forall x \in G,\; (x^{-1})^{-1} = x$
    \item $\forall x, y \in G, \; (xy)^{-1}=y^{-1}x^{-1}$
\end{enumerate}

\prop Let $G$ be a group. 
\begin{enumerate}
    \item If $x, y \in G$ s.t. $xy = e$, then $x=y^{-1}$ and $y=x^{-1}$
    \item If $x, g \in G$ s.t. $xg = x$ (or $gx = x$), then $g = e$
\end{enumerate}

\pagebreak
\prop (cancellation laws) Let $G$ be a group, and $x, y, z \in G$. Then 
\begin{enumerate}
    \item If $xy = xz$, then $y = z$
    \item If $yx = zx$, then $y = z$
\end{enumerate}

\bigskip 

\noindent 
\textbf{Corollary:} Let $G$ be a group with $g \in G$. Define $f_1, f_2: G \rightarrow G$ by $f_1(x) = gx,\; f_2(x) = xg$. Then $f_1$ and $f_2$ are 1 to 1 and onto.

\bigskip 

\noindent 
\textbf{Corollary:} (stated differently) Let $G$ be a group, $g \in G$. Then $\forall y \in G$,
\begin{enumerate}
    \item $\exists! x_1 \in G$ s.t. $gx_1 = y$
    \item $\exists! x_2 \in G$ s.t. $x_2g = y$
\end{enumerate} 

\thm Let $G$ be a cyclic group. Then $G$ is abelian.

\thm Let $G$ be a group, $a \in G$. Then $o(a) = |\langle a \rangle|$

\bigskip 

\noindent 
\textbf{Lemma:} Let $G$ be a group, $x \in G, \; m, n \in \Z$. Suppose $o(x) = n \ge 1$ and $x^m = e$. Then $n \mid m$

\prop Let $G$ be a group, with $a \in G$. 
\begin{enumerate}
    \item $o(a) = \infty \Leftrightarrow (\forall j, k \in \Z, \; a^j=a^k \text{ iff } j=k$)
    \item $o(a) = n \Leftrightarrow (\forall j, k \in \Z, \; a^i=a^j \text{ iff } \congmod{j}{k}{n}$
\end{enumerate}

\thm Let $m, n \in \Z$, not both zero. Then $\exists x, y \in \Z$ s.t. $mx + ny = (m, n)$.

\thm Let $G$ be a group, $x \in G$, $m, n \in \Z$ with $n \ge 1$. Then 
\begin{enumerate}
    \item $o(x) = o(x^{-1})$
    \item if $o(x) = n$ and $x^m = e$, then $n \mid m$
    \item if $o(x) = n$, then $o(x^m) = \frac{n}{\gcd(m, n)}$
\end{enumerate}

\pagebreak
\thm Let $G$ be a group, $H \subseteq G$ a subset. Then the following are equivalent:
\begin{enumerate}
    \item $H$ is a subgroup of $G$
    \item $H$ satisfies: \begin{enumerate}
        \item $e \in H$
        \item $H$ is closed under $\ast$
        \item $H$ is closed under inverses.
    \end{enumerate}
\end{enumerate}

\thm Let $G$ be a group, with $a \in G$. Then $\langle a \rangle$ is a subgroup of $G$. 

\thm Let $G$ be a group, with $H, K \subseteq G$ subgroups. Then $H \cap K$ is a subgroup of $G$.

\thm Let $G$ be a cyclic group and $H \subseteq G$ a subgroup. Then $H$ is also cyclic.

\thm (5.7) If $G$ is infinite cyclic, its subgroups are $\langle0\rangle = \{0\}, \langle1\rangle = \langle-1\rangle = \Z, \langle3\rangle = \langle-3\rangle, \ldots$

\thm (5.5) Let $G = \langle a \rangle$ cyclic of order $n$ (with $1 \le n < \infty)$
\begin{enumerate}
    \item $\forall m \ge 1$, $G$ has a subgroup $H$ with $|H| = m \Longleftrightarrow m \mid n$
    \item In that case, that subgroup $H$ is unique
    \item $\forall r, s \in \Z$, we have $\langle a^r \rangle = \langle a^s \rangle \Longleftrightarrow (r, n) = (s, n)$
\end{enumerate}

\noindent
\textbf{Corollary:} (5.6) The subgroups of $C_n$ are, for each $k \ge 1$ with $k \mid n$, $\langle \frac{n}{k} \rangle$