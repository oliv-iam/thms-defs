\noindent
\textbf{\LARGE Definitions and Theorems} \\
\large 2-5: Basic properties of groups

\normalsize

\section*{Definitions}

\defn $\ast$ is a \textit{binary operation} on a set $G$ if $\forall x, y \in G,\; x \ast y \in G$

\bigskip 

\defn Let $G$ be a set, and let $\ast: G \times G \rightarrow G$ be a \textit{binary operation} on $G$. Suppose
\begin{enumerate}
    \item $\ast$ is \textit{associative}, i.e., $\forall x, y \in G, \; (x \ast y) \ast z = x \ast (y \ast z)$
    \item $\ast$ has an \textit{identity element}, i.e., $\exists e \in G$ s.t. $\forall x \in G, \; x \ast e = e \ast x = x$
    \item $\ast$ has \textit{inverses}, i.e., $\forall x \in G, \; \exists y \in G $ s.t. $x \ast y = y \ast x = e$
\end{enumerate}
We then say that $(G, \ast)$ is a \textit{group}.
If $(G, \ast)$ is a group and $\ast$ is \textit{commutative}, we say $G$ is an \textit{abelian group}.

\section*{Theorems}
\textbf{Lemma:} (division algorithm) Let $a, n \in \Z$ with $ n \ge 1$. Then $\exists ! q, r \in \Z$ s.t.
\begin{enumerate}
    \item $a = qn + r$
    \item $0 \le r \le n - 1$
\end{enumerate}

\prop Let $G$ be a group.
\begin{enumerate}
    \item The identity element $e$ is unique
    \item $\forall x \in G$, the inverse $x^{-1}$ is unique
\end{enumerate}

\prop Let $G$ be a group. Then 
\begin{enumerate}
    \item $\forall x \in G,\; (x^{-1})^{-1} = x$
    \item $\forall x, y \in G, \; (xy)^{-1}=y^{-1}x^{-1}$
\end{enumerate}

\prop Let $G$ be a group. 
\begin{enumerate}
    \item If $x, y \in G$ s.t. $xy = e$, then $x=y^{-1}$ and $y=x^{-1}$
    \item If $x, g \in G$ s.t. $xg = x$ (or $gx = x$), then $g = e$
\end{enumerate}

\prop (cancellation laws) Let $G$ be a group, and $x, y, z \in G$. Then 
\begin{enumerate}
    \item If $xy = xz$, then $y = z$
    \item If $yx = zx$, then $y = z$
\end{enumerate}

\bigskip 

\noindent 
\textbf{Corollary:} Let $G$ be a group with $g \in G$. Define $f_1, f_2: G \rightarrow G$ by $f_1(x) = gx,\; f_2(x) = xg$. Then $f_1$ and $f_2$ are 1 to 1 and onto.

\bigskip 

\noindent 
\textbf{Corollary:} (stated differently) Let $G$ be a group, $g \in G$. Then $\forall y \in G$,
\begin{enumerate}
    \item $\exists! x_1 \in G$ s.t. $gx_1 = y$
    \item $\exists! x_2 \in G$ s.t. $x_2g = y$
\end{enumerate} 