\noindent
\textbf{\LARGE Definitions and Theorems} \\
\large 2-5: Basic properties of groups

\normalsize

\section*{Definitions}

\defn Let $G$ be a set, and let $\ast: G \times G \rightarrow G$ be a \textit{binary operation} on $G$. Suppose
\begin{enumerate}
    \item $\ast$ is \textit{associative}, i.e., $\forall x, y \in G, \; (x \ast y) \ast z = x \ast (y \ast z)$
    \item $\ast$ has an \textit{identity element}, i.e., $\exists e \in G$ s.t. $\forall x \in G, \; x \ast e = e \ast x = x$
    \item $\ast$ has \textit{inverses}, i.e., $\forall x \in G, \; \exists y \in G $ s.t. $x \ast y = y \ast x = e$
\end{enumerate}
We then say that $(G, \ast)$ is a \textit{group}.
If $(G, \ast)$ is a group and $\ast$ is \textit{commutative}, we say $G$ is an \textit{abelian group}.