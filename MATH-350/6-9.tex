\noindent
\textbf{\LARGE Definitions and Theorems} \\
\large 6-9: Symmetric groups, direct products, cosets 

\normalsize 

\section*{Definitions}
\defn Let $G, H$ groups. The set 
\[G \times H = \{(g, h) \mid g \in G \text{ and } h \in H\}\] 
with operation * given by $(g_1, h_1) \ast (g_2, h_2) = (g_1g_2, h_1h_2)$
is called the \textit{(direct) product} of $G$ and $H$.

\bigskip
\noindent
\textbf{Def/Prop:} Let $G_1, \ldots, G_n$ be groups. The set 
\[G_1 \times \ldots \times G_n = \{(g_1, \ldots, g_n) \mid g_i \in G_i, \; \forall i = 1, \ldots, n\}\]
with the operation $(g_1, \ldots, g_n)(\tilde{g}_1, \ldots, \tilde{g}_n) = (g_1\tilde{g}_1, \ldots, g_n\tilde{g}_n)$
is a group, called the \textit{(direct) product group} of $G_1, \ldots, G_n$.

\subsection*{Related Definitions}
\defn Let $m_1, \ldots, m_n \in \Z$, not all zero. Then lcm($m_1, \ldots, m_n$) is the \textit{least common multiple} of $m_1, \ldots, m_n$ is the smallest positive integer $M \ge 1$ s.t. $m_i \mid M \forall i = 1, \ldots, n$.\\
Note: lcm($m,n$) = $\frac{mn}{\gcd(m,n)}$


\section*{Theorems}
\prop Let $G, H$ be groups. Then $G \times H$ (with the above operation) is a group.

\thm Let $A \subseteq G, B \subseteq H$ subgroups. Then $A \times B \subseteq G \times H$ is a subgroup.

\prop Let $G_1, \ldots, G_n$ be groups. Let $G = G_1 \times \ldots \times G_n$. Then $G$ is abelian iff all of $G_1, \ldots, G_n$ is abelian.

\thm Let $G_1, \ldots, G_n$ groups. Let $g_i \in G_i$ for $i = 1, \ldots, n$ with order $o(g_i)$. Then 
$$o((g_1, \ldots, g_n)) = 
\begin{cases}
    \infty & \text{if any } o(g_i) \text{ is } \infty \\
    \text{lcm}(o(g_i), \ldots, o(g_n)) & \text{otherwise}
\end{cases}$$