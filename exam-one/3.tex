\noindent
\textbf{\LARGE Definitions and Theorems} \\
\Large Week: Sep. 16 - 20

\section{Theorems}

\textbf{Thm 1.3.4+:} Let $S$ be a subset of a vector space $V$. Then Span($S$) is a subspace of $V$. In fact, Span($S$) is the smallest subspace of $V$ that contains $S$, i.e., if $W \subseteq V$ is any subspace of $V$ s.t. $S \subseteq W$, then Span($S$) $\subseteq W$.

\bigskip

\noindent
\textbf{Thm:} Let $S$ be a subset of a vector space $V$. Then $S$ is a subspace of $V$ iff $S =$ Span($S$).

\bigskip

\noindent
\textbf{Thm:} If $W_1$ and $W_2$ are subspaces of a vector space $V$, then $W_1 \cap W_2$ is also a subspace of $V$.

\bigskip

\noindent
\textbf{Thm:} Let $W_1 , W_2$ be subspaces of a vector space $V$. \\
Then $W_1 + W_2 = $ Span($W_1 \cup W_2$). Consequentially, $W_1 + W_2$ is a subspace of $V$. In fact, $W_1 + W_2$ is the smallest subspace that contains $W_1 \cup W_2$.

\bigskip

\noindent
\textbf{Thm:} Let $V$ be a vector space and let $\bar x \in V$. Then $S := \{\bar x \}$ is linearly independent iff $\bar x \ne \bar 0_v$.

\pagebreak
\section{Propositions}

\textbf{Prop:} Let $S$ be a subset of a vector space $V$. Then $S \subseteq $ Span($S$) $\subseteq V$.

\section{Lemmas} 

\textbf{Lemma:} Let $S_1 , S_2$ be non empty subsets of a vector space $V$. Then Span($S_1 \cup S_2$) = Span($S_1$) $ + $ Span($S_2$)

\pagebreak
\section{Definitions}

\textbf{Def:} Let $V$ be a vector space and $W_1, W_2$ be two subspaces of $V$. The sum of $W_1$ and $W_2$ is $W_1 + W_2 :=$ \{$\bar w_1 + \bar w_2 \mid \bar w_1 \in W_1$ and $\bar w_2 \in W_2$\}.

\bigskip

\noindent
\textbf{Def:} Let $A, B$ sets. Define
\begin{enumerate}
    \item \textit{Intersection} of $A$ and $B$ is $A \cap B := \{ x \mid x \in A$ and $x \in B \}$.
    \item \textit{Union} of $A$ and $B$ is $A \cup B := $ \{$x \mid x \in A$ or $x \in B$\}.
\end{enumerate}

\bigskip

\noindent
\textbf{Def 1.4.2/1.4.4:} Let $V$ be a vector space. A set of vectors $S := \{ \bar v_1 , ... , \bar v_n \} \subseteq V$ is said to be 
\begin{enumerate}
    \item \textit{Linearly dependent} if there exists constants $ c_1, ..., c_n \in \mathbb{R} $, not all zero, such that $c_1 \bar v_1 + c_2 \bar v_2 + ... + c_n \bar v_n = 0$.
    \item \textit{Linearly independent} if $S$ is not linearly dependent, i.e., if the only constants $c_1, ..., c_n \in \mathbb{R}$ that make the equation $c_1 \bar v_1 + ... + c_n \bar v_n = \bar 0$ true are $c_1 = c_2 = ... = c_n = 0$.
\end{enumerate}

